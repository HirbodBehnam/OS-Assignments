\smalltitle{سوال 1}
\\\noindent
در این سوال فرض می‌کنیم که تابع
\lr{compare and swap}
مانند اسلاید‌ها به صورت زیر پیاده سازی شده است:
\codesample{codes/cas.c}
همچنین فرض می‌کنیم که تابع‌های
\lr{compare and swap}
و
\lr{test and set}
به صورت
\lr{atomic}
و
\lr{uninterruptible}
پیاده‌سازی شده‌اند.
\begin{enumerate}
    \item خیر ددلاکی نداریم و همچنین در تابع دوم لاکی نیز وجود ندارد!
    دقت کنید که در صورتی که مقدار
    \codeword{var1}
    برابر 1 باشد، با اینکه هیچ
    \lr{swap}ای
    انجام نمی‌شود ولی از حلقه خارج می‌شویم چرا که مقدار قدیمی
    \codeword{var1}
    برابر 1 بود و در نتیجه شرط حلقه
    \codeword{false}
    می‌شود! برای درست کردن این مشکل کافی است که
    \codeword{!}
    قبل از تابع را برداریم.
    با فرض وجود
    \lr{not}،
    جواب جفت سوالات الف و ب خیر است چرا که همان طور که گفته شد اصلا لاکی وجود ندارد.
    لازم به ذکر است زمانی که
    \codeword{var1}
    برابر 0 است نیز، در دور بعدی حلقه از حلقه خارج می‌شویم و در آن گیر نمی‌کنیم.

    \noindent
    اما از طرفی دیگر در صورتی که آن
    \codeword{!}
    را برداریم، در صورتی که از زمانبند
    \lr{non-preemptive}
    استفاده می‌کنیم، مشکلی پیش نمی‌آید چرا که هر کدام از توابع برای خودشان لاک را می‌گیرند و نگه می‌دارند
    و کارشان را انجام می‌دهد و لاک را در نهایت آزاد می‌کنند.
    \item از طرفی دیگر در صورتی که
    \lr{preemptive scheduler}
    را استفاده کنیم ممکن است که به ددلاک بر بخوریم. بدین صورت که فرض کنید که تابع
    \codeword{function1}
    خط اولش اجرا می‌شود و
    \codeword{var1}
    برابر 1 می‌شود. از طرفی دیگر نیز فرض کنید که بلافاصه
    \lr{context switch}
    اتفاق می‌افتد و تابع دوم شروع به کار می‌کند. در این تابع
    \codeword{var2}
    را برابر 1 می‌کنیم و در لوپی قرار می‌گیریم که سعی می‌کند که زمانی که
    \codeword{var1}
    برابر 0 بود، آنرا برابر 1 کند و از تابع خارج شود.
    از طرفی دیگر تابع اول نیز اینقدر
    \codeword{var2}
    را برابر یک قرار می‌دهد تا زمانی که مقدار اولیه‌ی آن برابر 0 باشد. پس در اینجا ددلاک رخ می‌دهد.
    \item اگر منظور الگوریتم‌های \lr{deadlock prevention} باشد
     نه لزوما. همان طور که در کلاس نیز گفته شد 4 شرط وجود دارد که حتی اگر یکی از آنها نقض شود می‌توان
    از ددلاک جلوگیری کرد. به عنوان مثال کافی است که بگوییم یک پراسس نمی‌تواند بیشتر از یک لاک را نگه دارد.
    در این حالت دیگر ددلاکی به وجود نمی‌آید. یا به عنوان مثال جاب‌ها را به صورت
    \lr{non preemptive}
    اجرا کنیم که
    \lr{mutual exclusion}
    از بین برود.

    اما از طرفی دیگر اگر منظور برای الگوریتم‌های
    \lr{deadlock avoidance}
    باشد، این موضوع درست است. چرا که در هر لحظه باید تمامی اطلاعات پراسس‌ها را در نظر داشته باشیم که
    ببینیم چه کسی چه ریسورس‌هایی را درگیر کرده است که در صورت نیاز تصمیم بگیریم که آیا می‌توانیم منبعی را
    به پردازه‌ای تخصیص دهیم یا خیر.
    \item به صورت کلی الگوریتم‌های
    \lr{non preemptive}
    این مشکل را حل می‌کنند. این الگوریتم‌ها صبر می‌کنند که تا کار یک
    \lr{job}
    به صورت کامل تمام شود و پس آن سراغ کار بعدی می‌روند. پس هر منبعی که هر پراسس لاک کرده بود
    حتما آنلاک می‌شود چرا که اصلا پراسس تمام می‌شود قبل از اینکه
    \lr{CPU}
    دست کسی دیگر قرار بگیرد. این الگوریتم‌ها عبارت‌اند از الگوریتم‌هایی همچون
    \lr{FIFO}
    یا
    \lr{SJF}.
\end{enumerate}



