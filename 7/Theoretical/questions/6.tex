\smalltitle{سوال 6}
\\\noindent
در این سیستم فرض می‌کنیم که دیتا به صورت بیت به بیت خوانده و نوشته می‌شود. یعنی پردازه‌‌ها نمی‌توانند منتظر
بمانند که
"حداقل $n$ بیت دیتا آماده باشد و بعد آنرا بخوانند"
و دیتا را به صورت یک بیت استریم دریافت می‌کنند.
(عملا فرض کردیم اندازه‌ی هر بلوک یک بیت است.)

\noindent
حال ثابت می‌کنیم ممکن است که به ددلاک بخوریم. فرض کنید که
$P_1$
در هر ثانیه یک رشته‌ی 1 بایتی در
$B_{12}$
می‌نویسید.
پردازه‌ی
$P_2$
نیز هر بار یک بایت از
$B_{12}$
می‌خواند و آنرا دوبار در
$B_{23}$
می‌نویسید. حال فرض کنید که دیسک به صورت کلی 3 بایت جا دارد. در ابتدا
$P_1$
بایت اول را در دیسک می‌نویسد و این بایت در توسط
$P_2$
خوانده می‌شود. قبل از اینکه
$P_2$
نتیجه‌ی خروجی‌اش را در
$B_{23}$
بنوسید، فرض کنید که
$P_1$
سه بایت در دیسک می‌نویسد و دیسک پر می‌شود.
پس در اینجا ددلاک رخ می‌دهد چرا که دیگر نه
$P_1$
می‌تواند دیتای بیشتری را بنویسید و نه
$P_2$
می‌تواند دیتایی که قرار بود در بافر بنویسید را در دیسک بنویسد.

\noindent
برای گرسنگی فرض کنید که
$P_2$
بدون توجه به ورودی شروع می‌کند و تک بیت تک بیت در
$B_{23}$
می‌نویسید.
$P_3$
نیز دقیقا با همان سرعت از
$B_{23}$
می‌خواند. فرض کنید که در عوض
$P_1$
دو بیت دو بیت در
$B_{12}$
می‌نویسید. حال فرض کنید که در ابتدا
$P_1$
دو بیت در
$B_{12}$
می‌نویسید. همزمان با آن
$P_2$
نیز یک بیت در خروجی می‌نویسید. پس بافر پر می‌شود. در همین حین،
$P_3$
آن بیت را بر می‌دارد از بافر. ولی همچنان
$P_1$
منتظر است که بافر خالی شود که بتواند داده‌ی خود را بنویسد. پس
$P_1$، starve
می‌شود. چرا که دوباره
$P_2$
آن بافر خالی را پر می‌کند و
$P_3$
آنرا می‌خواند.