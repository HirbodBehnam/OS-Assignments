\smalltitle{سوال 11}
\\\noindent
\link{http://web.cs.ucla.edu/classes/winter16/cs111/scribe/11b/index.html}{منبع 1}
\link{https://en.wikipedia.org/wiki/Journaling\_file\_system}{منبع 2}
\begin{enumerate}
    \item در فایل سیستم‌های ژورنالی مثل
    \lr{NTFS} یا \lr{EXT4}
    یک قسمتی از دیسک به چیزی به اسم ژورنال تخصیص می‌یابد. در این ژورنال عملیات نوشتن ذخیره می‌شود که
    فعلا در فایل سیستم تاثیر خود را نگذاشته‌اند. در آینده این ژورنال تغییراتش در فایل سیستم اعمال می‌شود.
    این ژورنال کمک می‌کند که درصورت مواردی چون قطعی برق، فایل سیستم خراب نشود.
    \item \lr{Internal fragmentation} بدین معنا است که یک فایل به قدری کوچک است که
    تنها در یک
    \lr{sector}
    ذخیره می‌شود. زمانی که اندازه‌ی این فایل از این سکتور کمتر باشد،
    \lr{internal fragmentation}
    رخ می‌دهد. از طرفی دیگر
    \lr{external fragmentation}
    چیزی است که به خاطر آن پیشنهاد می‌شود که ما هر چند وقت یک‌بار دیسک‌هایمان را
    \lr{defragment}
    کنیم. بدین معنی که تیکه‌های فایلمان در جا‌های مختلف هارد ذخیره شده است. این موضوع در دیسک‌هایی مانند
    \lr{SSD} یا \lr{NVMe}
    اهمیتی ندارد. اما از طرفی در دیسک‌های مکانیکی این پراکندگی می‌تواند سرعت خواندن و نوشتن را بشدت کاهش دهد.
    \item عملا کاری که این سیستم می‌کند این است که فایلی که در دیسک است را در مموری می‌آورد.
    این کار باعث می‌شود که خواندن فایل بشدت سریعتر شود. اما از طرفی دیگر نوشتن ممکن است که با چالش‌هایی همراه
    باشد که مشابه آنرا در
    \lr{dirty page}ها و کش خود
    \lr{CPU}
    در معماری دیدیم.
\end{enumerate}