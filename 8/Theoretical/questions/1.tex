\smalltitle{سوال 1}
\begin{enumerate}
    \item در ابتدا مموری را اینقدر نصف می‌کنیم تا اینکه به بلاکی برسیم که اگر آنرا نصف کنیم کمتر از
    \lr{2KB}
    شود. این بلاک در اینجا همان
    \lr{2KB}
    است. سپس اینقدر بالا می‌آییم تا به بلاک
    \lr{16KB}
    برسیم. این بلاک را برای درخواست
    \lr{13KB}
    رزرو می‌کنیم.
    در ادامه بلاک کنار
    \lr{2KB}
    را اینقدر نصف می‌کنیم تا به 128 بایت برسیم. این بلاک را برای درخواست
    128 بایتی مصرف می‌کنیم.
    برای درخواست آخر نیز اینقدر بالا میرویم تا به بلاک
    32 کلیوبایتی برسیم.
    \item در ابتدا برای تخصیص اول به $\log_2{\frac{128 KB}{2 KB}} = \log_2{64} = 6$
    نصف کردن نیاز داریم.
    سپس برای درخواست دوم از
    یک بلاک
    \lr{16KB}
    باقی مانده استفاده می‌کنیم. پس نصف کردنی صورت نمی‌گیرد.
    برای تخصیص سوم نیاز به
    $\log_2{\frac{2 KB}{128 B}} = 5$
    تقسیم کردن داریم. برای بلاک آخر نیز از یک بلاک 32 کلیوبایتی موجود استفاده می‌کنیم.
    \item دقیقا همین 5 تقسیم که در قسمت قبل بدست آوردیم. دقیقا همین مقدار coelesce کردن نیاز است.
    \item اصلا نیازی به \lr{split} کردن نداریم.
    چرا که همیشه بلاکی خالی است. به ترتیب بلاک‌هایی با این سایز اساین می‌شود:
    \begin{gather*}
        2KB, 16KB, 128B, 32KB
    \end{gather*}
    \item در این قسمت نیز نیازی به coelesce نداریم. چرا که اصلا split نداشتیم.
\end{enumerate}


