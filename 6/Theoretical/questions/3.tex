\smalltitle{سوال 3}
\\\smalltitle{قسمت اول}
\\\noindent
این کد
\lr{mutual executation}
را حل می‌کند.
در ابتدا قسمت دوم سوال را جواب
می‌دهیم. فرض کنید که 3 ترد 1 تا 3 در حال اجرا شدن هستند و
$turn = 1$
است و تمامی
\lr{flag}ها
برابر 0 هستند.
فرض کنید که کلا هیچ
\lr{time slice}ای
به ترد شماره 1 نمی‌رسید.
ترد 2 و 3 همزمان وارد حلقه‌ی اول
\lr{while}
می‌شوند. همزمان متوجه می‌شوند که
\lr{turn}
برابر شماره خودشان نیست و همزمان متوجه می‌شوند که
$flag[1] == 0$
است. پس هر دو به خط
$turn = i$
می‌رسند. حال فرض کنید که
\lr{time slice}
ترد شماره 2 تموم می‌شود ولی ترد 3 ادامه پیدا می‌کند. ترد 3
\lr{turn}
را برابر 3 می‌کند و به شرط حلقه می‌رود. از آنجا که
\lr{turn}
برابر 3 است، کد ادامه پیدا می‌کند و به خط
$flag[i] = 2$
می‌رسد. سپس کار ترد 2 از سر گرفته می‌شود. این ترد نیز
\lr{turn}
را برابر 2 قرار می‌دهد و شرط حلقه را چک می‌کند و متوجه می‌شود که برابر 2 هست. پس این ترد نیز به خط
$flag[i] = 2$
می‌رسد. پس تا اینجا فهمیدیم که این تیکه کد به تنهایی مشکل
\lr{mutual executation}
را حل نمی‌کند (جواب سوال دوم).

\noindent
حال اثبات می‌کنیم که قسمت دوم مشکل را حل می‌کند. فرض کنید که دو ترد همزمان وارد
\lr{critical section}
شده‌اند. این نشان می‌دهد که همزمان دو ترد باید از
\codeword{for}
بالای آن رد شده باشند. یعنی اینکه دو ترد باید تشخیص داده باشند که هیچ تردی غیر از خودشان
\lr{flag}
خودش را برابر 2 نکرده است. اما این موضوع امکان پذیر نیست. دلیل این موضوع این است که به محض اینکه
یک ترد از خط
$flag[i] = 2$
عبور می‌کنند، اگر هر ترد دیگری به این خط برسد، اثبات می‌کنیم که حتی اگر در حال اجرای حلقه نیز باشیم،
حداقل یک ترد به
$L$
می‌رود.

\noindent
اگر تردی در
\lr{critical section}
باشد که ترد دیگر نمی‌تواند از
\codeword{for}
بیرون بیاید چرا که سر تردی که در
\lr{critical section}
است، به دستور
\codeword{goto}
می‌خورد. از طرفی فرض کنید که دو ترد همزمان در
\codeword{for}
هستند. اگر تردی فلگ را وسط کار برابر 2 قرار دهد، دو اتفاق می‌تواند بیفتد. فرض کنید که در آینده قرار است
که تردی که از قبل اجرا می‌شد به چک کردن فلگ آن یکی ترد برسد؛ در این صورت پس در آینده این ترد به
\codeword{goto}
می‌رسد و از حلقه خارج می‌شود. ممکن است که ترد دوم وارد
\lr{ciritical section}
شود که مهم نیست. (یک ترد وارد این قسمت شده است!)
از طرفی دیگر فرض کنید که شماره ترد دوم کمتر از شماره ترد اول باشد و در گذشته ترد اول
چک کرده بود که این ترد فلگش برابر 2 نیست. در این حالت، ترد دوم متوجه می‌شود که ترد اولی
فلگش برابر 2 است و پس وارد
\lr{critical section}
نمی‌شود.

\smalltitle{قسمت دوم}
\\\noindent
به قسمت اول مراجعه شود. در آنجا توضیح داده شده است که جواب خیر است.

