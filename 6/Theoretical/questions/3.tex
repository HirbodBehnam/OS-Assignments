\smalltitle{سوال 3}
\\\smalltitle{قسمت اول}
\\\noindent
این کد
\lr{mutual executation}
را حل نمی‌کند.
البته با یک تغییر خیلی کوچیک حل می‌کند ولی کد داخل سوال خیر. در ابتدا قسمت دوم سوال را جواب
می‌دهیم. فرض کنید که 3 ترد 1 تا 3 در حال اجرا شدن هستند و
$turn = 1$
است و تمامی
\lr{flag}ها
برابر 0 هستند.
فرض کنید که کلا هیچ
\lr{time slice}ای
به ترد شماره 1 نمی‌رسید.
ترد 2 و 3 همزمان وارد حلقه‌ی اول
\lr{while}
می‌شوند. همزمان متوجه می‌شوند که
\lr{turn}
برابر شماره خودشان نیست و همزمان متوجه می‌شوند که
$flag[1] == 0$
است. پس هر دو به خط
$turn = i$
می‌رسند. حال فرض کنید که
\lr{time slice}
ترد شماره 2 تموم می‌شود ولی ترد 3 ادامه پیدا می‌کند. ترد 3
\lr{turn}
را برابر 3 می‌کند و به شرط حلقه می‌رود. از آنجا که
\lr{turn}
برابر 3 است، کد ادامه پیدا می‌کند و به خط
$flag[i] = 2$
می‌رسد. سپس کار ترد 2 از سر گرفته می‌شود. این ترد نیز
\lr{turn}
را برابر 2 قرار می‌دهد و شرط حلقه را چک می‌کند و متوجه می‌شود که برابر 2 هست. پس این ترد نیز به خط
$flag[i] = 2$
می‌رسد. پس تا اینجا فهمیدیم که این تیکه کد به تنهایی مشکل
\lr{mutual executation}
را حل نمی‌کند (جواب سوال دوم).

\noindent
حال به تحلیل قسمت دوم می‌رسیم.
فرض کنید که ترد شماره 3 در ابتدا شروع به اجرا کردن کد حلقه می‌کند. این ترد، فقط فلگ‌های کمتر از 3
را چک می‌کند. دقت کنید که به خاطر شرط حلقه این اتفاق می‌افتد. چرا که به محض اینکه
$j == i$
می‌شود، شرط حلقه نقض می‌شود و کلا حلقه متوقف می‌شود. این یعنی اگر شرط حلقه‌ را صرفا
$j < i$
می‌کردیم هیچ فرقی با این شرط نداشت. پس کاری که ترد 3 می‌کند این است که چک می‌کند که
$flag[1]$
و
$flag[2]$
برابر 2 نباشند. از آنجا که فرض کردیم که ترد یک کلا اجرا نمی‌شود که مشکلی وجود ندارد و فرض کردیم که فعلا
ترد شماره 2 سر خط
$flag[i] = 2$
نگه داشته شده است. پس ترد 3 وارد
\lr{critical section}
می‌شود. حال فرض کنید که کار ترد 2 نیز ادامه پیدا می‌کند. این ترد فقط ترد 1 را چک می‌کند که
\lr{flag}
آن برابر 2 نباشد که در این مثال نیست! پس ترد دو نیز وارد
\lr{critical section}
می‌شود.

\noindent
اما با یک تغییر خیلی ریز می‌توان مشکل را حل کرد. کافی است که کد را به صورت زیر تغییر دهیم:
\codesample{code/3.c}
\noindent
تنها تغییر کد خط 10 تا 13 است. اول از همه شرط حلقه تغییر کرد به
$j < n$
که
$n$
تعداد کل ترد‌ها است. این کد به این دلیل درست کار می‌کند که اگر حتی یک ترد به
$flag[i] = 2$
رسیده باشد، کل لاک ریست می‌شود.

\smalltitle{قسمت دوم}
به قسمت اول مراجعه شود. در آنجا توضیح داده شده است که جواب خیر است.

