\smalltitle{سوال 1}
\\\noindent
از سه
\lr{semaphore}
استفاده می‌کنیم. یک سمافور صرفا مسئول درست کردن صف دانشجوها است. این سمافور را
\linebreak
\codeword{drink\_queue\_sem}
می‌نامیم. صرفا زمانی که هر مشتری می‌خواهد نوشیدنی را بردارد، بر روی این سمافور
\lr{wait}
می‌کنیم. در ابتدا این سمافور مقدار 1 را دارد. سمافور بعدی
\codeword{bartender\_caller\_sem}
است. متصدی بستنی طرشت بر روی این سمافور همواره در حال
\lr{wait}
کردن است که یکی از دانشجو‌ها آن‌را صدا بزند. زمانی که این سمافور
\lr{signal}
شود متصدی بستنی ترشت دوباره نوشیدنی‌ها را بر روی پیشخوان قرار می‌دهد.
آخرین سمافور
\codeword{bartender\_done\_sem}
است. دانشجویی که متصدی را صدا کرده است، بر روی این سمافور
\lr{wait}
می‌کند تا متصدی نوشیدنی‌ها را پر کند و
\lr{signal}
کند. یک نمونه شبه کد با کتابخانه‌ی
\lr{pthreads}
را می‌توانید در زیر مشاهده کنید:
\codesample{code/1.c}
\noindent
در این کد چندین نکته وجود دارد که جا دارد به آن اشاره کنم.
در ابتدا از آنجا که تمامی دانشجو‌ها را به کمک سمافور
\linebreak
\codeword{drink\_queue\_sem}
به صف می‌کنیم، امکان ندارد که دوبار متصدی صدا زده شود. از طرفی دیگر نیز این موضوع نشان می‌دهد که
امکان ندارد که دو دانشجو در حال صبر کردن بر روی
\codeword{bartender\_done\_sem}
شود. از طرفی دیگر نیز در تعداد نوشیدنی‌ها
\lr{race}
وجود ندارد. چرا که همان طور که گفته شد یک دانشجو در اول صف است. زمانی هم که کد دانشجویی
که متصدی را صدا کرده بود به خط
17
(\codeword{drinks--})
می‌رسد، زمانی است که متصدی 
\codeword{bartender\_done\_sem}
را
\lr{signal}
کرده است و پس نوشیدنی‌ها حتما برابر
$M$
هستند.




