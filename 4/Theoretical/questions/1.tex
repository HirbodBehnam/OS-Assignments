\smalltitle{سوال 1}
\begin{enumerate}
    \item همان طور که سر کلاس نیز گفته شد
    \lr{simultaneous multithreading}
    یک روش است که بتوان از زمان مرده‌ای که یک هسته منتظر است که دیتا از مموری خوانده شود یا نوشته شود،
    یک ترد دیگر نیز اجرا کرد.
    معمولا طراحی این پردازنده‌ها به صورتی است که در یک کور آن چندین رجیستر فایل و یک
    \lr{ALU}
    قرار دارد.
    به کمک این طراحی زمانی که یکی از
    \lr{hardware thread}ها
    می‌خواهد که که به مموری دسترسی داشته باشد، همزمان که دیتا در حال خوانده شدن از مموری است،
    \lr{context switch}
    بین دو رجیستر فایل اتفاق می‌افتد و ترد دوم شروع به اجرا شدن می‌کند.
    در پردازنده‌های اینتل نام این تکنولوژی
    \link{https://en.wikipedia.org/wiki/Hyper-threading}{Hyper-threading}
    است.
    \item \lr{hardware thread}
    \item در تمامی پردازنده‌های
    \lr{Intel} و \lr{AMD}
    این نسبت 1 به 2 است.
    \item ممکن است که با طراحی درست \lr{CPU} سیستم‌عامل آنها را به صورت \lr{core}های
    جدا ببیند و بتواند بر روی آنها دستورالعمل اجرا کند. ولی در صورتی که سیستم‌عامل از این
    \lr{logical processor}ها
    آگاه باشد، آنگاه می‌تواند از کش استفاده‌ی بهتری بکند. به عنوان مثال احتمال اینکه دو ترد از یک برنامه
    به یک فضای مشترک حافظه دسترسی پیدا کنند زیاد است. پس منطقی است که این دو ترد را در یک کور و دو
    \lr{logical processor}
    جدا ران کنیم که بتوانیم حداکثر استفاده از
    \lr{L2 Cache}
    را ببریم.
    \\
    \link{https://stackoverflow.com/a/23937223/4213397}{منبع}
\end{enumerate}



