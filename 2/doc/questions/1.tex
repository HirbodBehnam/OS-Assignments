\smalltitle{سوال 1}\\
مزیتی که دارد این است که یک
interface
واحد برای کار با تمامی فایل‌ها و دستگاه‌ها به ما می‌دهد پس لازم نیست که توابع بیشتری را بلد باشیم.
از طرفی دیگر یکی از مشکلاتی که این روش دارد این است که به عنوان مثال سوکت‌ها و فایل‌ها و کیبورد‌ها صرفا با یک
عدد نشان داده می‌شوند و این ممکن است که باعث شود که پیدا کردن اینکه دقیقا نوشتن در این
\lr{file descriptor}
چه معنی دارد
(مثلا در سوکت نوشتن در کارت شبکه یا تکان دادن هد هارد برای نوشتن فایل)
زمان بر باشد.
همچنین یکی از مشکلاتی که پیش می‌آید این است که ممکن است در زمان کار با فایل بخواهیم از فلگ‌هایی استفاده
کنیم که در
\lr{socket}ها
موجود نیستند و برعکس. برای حل این مشکل توابع دیگری نیز وجود دارد که
\lr{file descriptor}های
به خصوصی را می‌گیرند و همچنین
\lr{flag}هایی
را نیز مخصوص آن می‌گیرند.
به عنوان مثال تابع
\link{https://man7.org/linux/man-pages/man2/recvmsg.2.html}{recv}
دقیقا همین
\lr{flag}ها
را دارد و در داکیومت آن آمده است که:

\begin{latin}
\noindent
    The only difference between recv() and read(2) is the presence of
    flags.  With a zero flags argument, recv() is generally
    equivalent to read(2).
\end{latin}








