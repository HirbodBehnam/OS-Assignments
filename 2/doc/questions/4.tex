\smalltitle{سوال 4}\\
با توجه به این
\link{https://academiccommons.columbia.edu/doi/10.7916/D8ZC89ZT/download}{PDF}
می‌توان نتیجه گرفت که خواسته‌ی سوال
\lr{Collapsing Layers}
است.

\noindent
با توجه به چیزی که در این
\lr{PDF}
می‌توان خواند هر لایه‌ی سیستم عامل به صورتی یک برنامه است که در حال ران شدن است. چرا که از لفظ
\lr{context switching}
استفاده شده است. این سیستم‌عامل زمانی که می‌خواهد از یک لایه‌ی بالاتر به لایه‌ی پایین تر دیتایی را بفرستد،
به صورت
\lr{force module}،
\lr{CPU}
در اختیار لایه‌ی پایین‌تر قرار می‌گیرد. همچنین این موضوع با موضوعی به اسم
\lr{Factoring Invariants}
ترکیب می‌شود که سرعت بیشتری داشته باشد. این طور که متوجه شدم این موضوع چیزی شبیه 
\lr{JIT (Just-in-time compilation)}
است که اجازه می‌دهد یک تابعی که با مقدار ثابت صدا می‌شود به نحوی
\lr{cache}
شود و این کار به زیاد شدن سرعت برخی توابع کمک می‌کند.

\noindent
به نظر من یکی از معایبی که این روش دارد پیچیده‌تر کردن سیستم‌عامل است. چرا که باید تمامی این
\lr{context switch}ها
به نحوی
\lr{hardcode}
شوند. از طرفی این موضوع نشان می‌دهد که برنامه‌های دیگر در زمان عوض کردن لایه‌ها نمی‌توانند
\lr{CPU}
را بدست بگیرند برای همین ممکن است که کمی سرعت برنامه‌های سمت
\lr{user}
کند شود.





