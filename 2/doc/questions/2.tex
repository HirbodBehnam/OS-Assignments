\smalltitle{سوال 2}\\
دو کاربرد مهم دارد:
یکی ارتباط برقرار کردن با سیستم‌عامل و فراخوانی
\lr{syscall}ها
و دیگری گرفتن سرعت بالا در جا‌هایی که به سرعت نیاز داریم.
دومین دلیل که مشخص است چرا که جاوا یک زبان است که به
bytecode
تبدیل می‌شود و توسط
JVM
آن اجرا می‌شود پس سرعتش از یک زبان کامپایل شده مثل
C
کمتر است. پس می‌توان قسمت‌های
\lr{time critical}
برنامه را با
C
و
\lr{C++}
نوشت که سرعت بالایی بگیریم.

\noindent
از طرفی دیگر ممکن است که در جایی نیاز داشته باشیم که یکی از دستور‌های سیستمی را فراخوانی کنیم
که مثلا در ویندوز وجود ندارد ولی در لینوکس دارد. در اینجا می‌توانیم مستقیم به کمک کد
\lr{C}،
\lr{syscall}ها
را فراخوانی کنیم.
یکی از مثال‌هایی که در اینترنت من پیدا کردم کتابخانه‌ی
\link{https://github.com/bbeaupain/nio\_uring}{nio\_uring}
است. این کتابخانه از دستور‌های سیستمی
\link{https://en.wikipedia.org/wiki/Io\_uring}{io\_uring}
در لینوکس نسخه‌ی
\lr{5.1}
به بعد استفاده می‌کند.
این توابع سیستمی به ما اجازه‌ی استفاده از
\lr{asynchronous I/O}
را می‌دهد که به صورت پیشفرض در جاوا وجود ندارد.