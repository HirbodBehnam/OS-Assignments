\smalltitle{سوال 3}\\
یکی از مشکلات این سیستم عامل این است که دسترسی کامل به برنامه‌های در حال اجرا می‌دهد. به عنوان مثال
یک برنامه‌ می‌تواند محتوایت هر جای رم را هر طور که می‌خواهد عوض کند. این موضوع به حمله کننده‌ها اجازه می‌دهد
که کد کرنل سیستم را
\lr{re-write}
کنند. همچنین برنامه‌ای سهوا می‌توانست با این کار سیستم را
\lr{soft lock}
کند. بدین منظور که نیاز به ری استارت کل سیستم باشد برای درست شدن سیستم عامل.
یکی دیگر از مشکلات این سیستم‌عامل نبود
\lr{memory allocator}
در نسخه‌ی 1 این سیستم عامل بود.
(\link{https://en.wikipedia.org/wiki/DOS\_API\#DOS\_INT\_21h\_services}{کد \lr{48h} در اینجا})
پس قبل از آن برنامه‌ها باید به صورت دستی حافظه را مدیریت می‌کردند و به تمام حافظه دسترسی داشتند.
در صورتی که می‌خواستیم دستورات
\lr{memory allocation}
را به صورتی اضافه کنیم به سیستم که جلوی دسترسی برنامه‌ها به تمام حافظه را بگیریم
(و فقط در قسمت \lr{allocate} شده در برنامه قابل دسترس باشد)
آنگاه تمامی برنامه‌های نوشته شده برای نسخه‌ی یک داس به مشکل می‌خوردند و
\lr{segfault}
می‌شدند.

\noindent
می‌توان وضعیت را با اضافه کردن
\lr{privileged mode}
در
\lr{CPU}
بهتر کرد. حداقل کاری که می‌توان کرد عدم اجازه دادن برنامه‌ها در
\lr{User Mode}
به تغییر یا خواندن مموری کرنل سیستم عامل است.







