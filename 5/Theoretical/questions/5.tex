\smalltitle{سوال 5}
\begin{enumerate}
    \item در این حالت که
    $q = 1$
    است، فرض می‌کنیم که در ابتدا تمام برنامه‌های
    \lr{IO queue}
    می‌شوند برای اجرا. به هر کدام از آنها
    \lr{1ms}
    زمان میرسد که در این زمان درخواست
    \lr{IO}
    خود را
    \lr{dispatch}
    می‌کنند. سپس پردازه‌ی
    \lr{CPU bound}
    نیز برای 1 میلی‌ثانیه اجرا می‌شود. زمانی که دوباره شروع می‌کنیم که اجرا کردن برنامه‌های
    \lr{IO bound}
    از آنجا که بیشتر از 10 میلی‌ثانیه گذشته است، درجا می‌توانند درخواست بعدی خود را آماده بکنند.
    پس بهره وری از پردازنده به صورت زیر است:
    \begin{gather*}
        \frac{11}{11 + 11 \times 0.1} = \frac{11}{12.1}
    \end{gather*}
    \item
    در صورتی که قرار باشد بعد از هر درخواست
    \lr{IO}
    پردازه را از
    \lr{CPU}
    خارج کنیم، (که قاعدتا این اتفاق می‌افتد)
    بهره وری به صورت زیر است:
    \begin{gather*}
        \frac{1 \times 10 + 10}{1 \times 10 + 10 + 11 \times 0.1} = \frac{20}{21.1}
    \end{gather*}
\end{enumerate}