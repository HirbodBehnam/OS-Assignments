\smalltitle{سوال 1}
\begin{itemize}
    \item \textbf{\lr{SJF}}:
    در ابتدا به کمک داده‌های مسئله و
    \lr{exponential averaging}
    مدت
    \lr{burst}های
    بعدی را در می‌آوریم.
    \begin{align*}
        \tau_{T_1} &= 0.5 \times 2 + 0.5 \times 4 = 3\\
        \tau_{T_2} &= 0.5 \times 4 + 0.5 \times 4 = 4\\
        \tau_{T_3} &= 0.5 \times 6 + 0.5 \times 6 = 6
    \end{align*}
    حال اگر بخواهیم که
    \lr{SJF}
    را اجرا کنیم، باید از پردازه‌ای شروع کنیم که کمترین
    \lr{burst time}
    پیش‌بینی شده را داشته باشد. پس در این مثال اول
    $T_1$
    وارد
    \lr{CPU}
    می‌شود که 4 واحد نیز کار آن طول می‌کشد. سپس پردازه‌ی
    $T_2$
    وارد می‌شود که کار آن 3 واحد طول می‌کشد. این این زمان نیز، کار
    $T_3$
    آماده می‌شود پس می‌تواند که درجا وارد
    \lr{CPU}
    شود که آن نیز 6 واحد زمانی طول می‌کشد.
    پس
    \lr{turnaround time}
    برابر است با
    $\frac{4 + (4 + 3) + (((4 + 3) + 6) - 3)}{3} = 21/3 = 7$
    \item \textbf{\lr{FCFS}}:
    فرض می‌کنیم پردازه‌ها به ترتیب شماره‌ی آنها قرار است که
    \lr{schedule}
    شوند. با این فرض ترتیب اجرا شدن پردازه‌ها دقیقا مثل
    \lr{SJF}
    می‌شود. پس
    \lr{turnaround time}
    برابر است با
    $\frac{4 + (4 + 3) + (((4 + 3) + 6) - 3)}{3} = 21/3 = 7$
    \item \textbf{\lr{RR}}: فرض می‌کنیم که
    $q = 1$
    است. با این فرض در ابتدا تنها بین پردازه‌های 1 و 2 باید جابه‌جا شویم چرا که پردازه‌ی 3 حاضر نیست.
    سپس بین تمام آنها جا به جا می‌شویم.
    در ابتدا یک واحد از
    $T_1$ و سپس یک واحد از $T_2$
    را انجام می‌دهیم. ولی از آنجا که
    $T_3$
    هنوز آماده نیست، پس دوباره به
    $T_1$
    بر می‌گردیم. ولی این بار وقتی که می‌خواهیم
    $T_3$
    را اجرا کنیم آماده است و اجرا می‌شود. جا به جا شدن بین پردازه‌ها به صورت زیر است:
    \begin{gather*}
        T_1 \rightarrow T_2 \rightarrow
        T_1 \rightarrow T_2 \rightarrow T_3 \rightarrow
        T_1 \rightarrow \textcolor{red}{T_2} \rightarrow T_3 \rightarrow
        \textcolor{red}{T_1} \rightarrow T_3
        \rightarrow T_3 \rightarrow T_3 \rightarrow \textcolor{red}{T_3}
        \\
        \text{Time Turnaround} = \frac{9 + 7 + (13 - 3)}{3} = \frac{26}{3}
    \end{gather*}
\end{itemize}



