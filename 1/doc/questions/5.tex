~\\\smalltitle{سوال 5}
\\\textbf{قسمت الف:}
یکی از خوبی‌های سیستم‌های
SMP
این است که تمامی کور‌ها می‌توانند درگیر برنامه‌های در حال اجرا شوند در صورتی که در
ASMP
حتما یک
CPU
کارش مدیریت
(master)
است که ممکن است در بازه‌ای بیکار باشد.
از طرفی دیگر ممکن است که در سیستم‌های
\lr{SMP}،
\lr{synchronization}
سخت باشد. در صورتی که در
\lr{ASMP}،
مستر می‌تواند تمام کور‌های دیگر را برای مدت کوتاهی متوقف کند و عملیات مورد نظر را انجام دهد.
\\\textbf{قسمت ب:}
چند پردازنده‌ها عملا توسط یک باس مشترک با همدیگر ارتباط برقرار می‌کنند و یک سیستم‌عامل
واحد بر روی آنها اجرا می‌شود و منابع مشترکی دارند.
در عوض کلاستر‌ها عملا کامپیوتر‌های جدایی هستند که باید به کمک شبکه با همدیگر ارتباط برقرار بکنند.