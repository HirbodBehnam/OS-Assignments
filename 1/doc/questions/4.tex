~\\\smalltitle{سوال 4}\\
یکی از سناریو‌هایی که چند
\lr{multi tasking}
برتری دارد زمانی است که به اشتباه برنامه‌ای در لوپ بی‌نهایت می‌افتد. در این حالت هیچ‌گاه
\lr{cpu}
در اختیار برنامه‌های دیگر قرار نمی‌گیرد.
برای برتری
\lr{multi programming}
فرض کنید که در حالت
\lr{multi tasking}
هر
$t$
واحد زمانی برنامه‌ها را از
cpu
بیرون می‌کشیم و برنامه‌ی دیگری را جای آن قرار می‌دهیم.
(\lr{context switching}).
فرض کنید که هر برنامه
$t+\epsilon$
واحد زمانی کار داشته باشد. در این حالت ما
\lr{process}ها
را دقیقا قبل از اینکه کارشان تمام شود از
\lr{cpu}
بیرون می‌کشیم. در صورتی که اگر اجازه می‌دادیم ذره‌ای بیشتر کار کنند کلا کار آن تمام می‌شد و لازم نبود که
آنرا در رم نگه داریم.