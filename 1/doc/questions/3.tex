~\\\smalltitle{سوال 3}
\\\textbf{قسمت الف:}
\lr{device controller}
به خود سخت افزار (به عنوان مثال هارد)
وصل است ولی
\lr{device driver}
عملا یک نرم افزار است که در سیستم‌عامل اجرا می‌شود.
وظیفه‌ی
\lr{device controller}
بافر کردن دیتا است و وظیفه‌ی
\lr{device driver}
خواندن آن دیتا و دادن آن به کرنل.
\\\textbf{قسمت ب:}
\lr{device controller}
به باس مادربورد وصل است و از آن طریق دیتا‌ها را جا به جا می‌کند. ممکن است که به کمک
\lr{DMA}
دیتا در رم ریخته شود یا اینکه
\lr{device driver}
دیتا را مستقیم از باس بخواند.
\\\textbf{قسمت ج:}
بعضی از درایور‌ها را می‌توان برای دستگاه‌های دیگر نیز استفاده کرد چرا که
\lr{device controller}
نیز می‌تواند کمی از این یکپارچه سازی را انجام دهد. ولی به صورت کلی بله لازم است که برای
تمامی دستگاه‌ها درایور‌های مختلف نوشت.